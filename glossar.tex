

% Glossar
\newglossaryentry{computer}
{
	name=computer,
	description={is a programmable machine that receives input,
		stores and manipulates data, and provides
		output in a useful format}
}
\newglossaryentry{foobar}{%
	name={Foobar},
	description={A strange animal, not to be confused}
}

\newglossaryentry{LERNBOX}{%
	name={LERNBOX},
	description={ Diese wurden in der regionalen Lehrkräftefortbildung multipliziert. Es handelt sich um spezifisch für diesen Unterrichtsgang entwickelte und erprobte Unterrichtsmaterialien. Bitte wenden Sie sich bei Rückfragen an die Fachberatung Chemie Ihres Staatlichen Schulamtes.}
}

%% Abkürzungen
\newacronym[description={Schülerübung oder Schülerversuch}]{SV}{SV}{Schülerübungen / Schülerversuch}

\newacronym[description={Lehrerdemonstrationsversuch}]{LD}{LD}{Lehrerdemonstrationsversuch}

\newacronym[description={Leitperspektive Verbraucherbildung}]{VB}{VB}{Verbraucherbildung}
	
\newacronym[description={Leitperspektive Prävention und Gesundheitsförderung}]{PG}{PG}{Prävention und Gesundheitsförderung}

\newacronym[description={Leitperspektive Berufsorientierung}]{BO}{BO}{Berufsorientierung}

\newacronym[description={Leitperspektive Medienbildung}]{MB}{MB}{Medienbildung}
	
\newacronym[description={Leitperspektive Bildung für nachhaltige Entwicklung}]{BNE}{bne}{Bildung für nachhaltige Entwicklung}
	
\newacronym[description={Fächerverbund BNT}]{BNT}{BNT}{Fachverweis; hier Fächerverbund BNT}

\newacronym[description={Fach Technik}]{T}{T}{Fachverweis; hier Fach Technik}	

\newacronym[description={Grund-Niveau}]{G}{g}{Grund-Niveau} 
	
\newacronym[description={Mittleres-Niveau}]{M}{m}{Mittleres-Niveau}

\newacronym[description={Erweitertes-Niveau}]{E}{e}{Erweitertes-Niveau}
