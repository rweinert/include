
\usepackage[a4paper,left=2.6cm, right=2.3cm,top=2.5cm, bottom=3.0cm]{geometry}

% input-Zeichensatz uft8
\usepackage[utf8]{inputenc}

% deutscher Zeichensatz
\usepackage[T1]{fontenc}

% Deutsche Silbentrennung
\usepackage[ngerman]{babel}

% Erweiterte Funktionen für Formeln / Mathematik
\usepackage{amsmath,amssymb,amstext}

% für Bilder und Graphiken
\usepackage{graphicx}
\graphicspath{Bilder/}

% Zur Darstellung von Webadressen, E-Mails usw.
% Besonders praktisch ist, dass man innerhalb von \url{} keine Maskierung verwenden muss:
% Statt '\_' verwendet man einfach '_'.
% http://ctan.org/pkg/url
\usepackage{url}

% Verweise 
\usepackage[colorlinks,linkcolor=blue]{hyperref}

\usepackage{csquotes}

% Glossar
\usepackage[acronym,nonumberlist,automake]{glossaries}
\makeglossaries
% % % % Verwendete Abkürzungen
\loadglsentries{include/glossar.tex}



% mehrseitige Tabellen
\usepackage{longtable} 

% Textumflossenen Bilder
\usepackage{wrapfig}

% Bildunterschriften linksbündig
\usepackage{caption, booktabs}
\usepackage{ragged2e} 
\setkomafont{caption}{\sffamily\fontsize{8}{9.5}\selectfont\RaggedRight} 
\setkomafont{captionlabel}{\usekomafont{caption}\bfseries} 
\setcapindent{0pt}

% package für die Seitenzahlen
\usepackage{scrpage2}